\chapter{Primal-Dual Interior Point Methods}\label{chapter:PDIPM}
The primal-dual interior point methods are proven to be a class of algorithms that solves a wide range of optimization problems including linear-programming-problems, convex-quadratic-programming-problems, semi-definite-programming etc.\cite{wright1997primal}. They fall under the category of interior point methods and employ the Newton's method to solve optimization problems in an efficient manner. The research works that gave rise to the field of \textit{Interior-Point Methods} and successively the \textit{Primal-Dual Methods} started with the publication of Karmarkar's paper \cite{karmarkar1984new} which marked an important point in the history of optimization algorithms. This chapter walks through the basic building blocks of the primal-dual interior point methods, its evolution as a path-following algorithm, the Mehrotra's predictor-corrector approach and finally looks at the complexity analysis of these algorithms and compare it with other common optimization algorithms like the simplex method.
\section{Newton's Method}\label{NM}

The Newton's method finds the roots of a function by moving along search directions generated by the the liner approximation of the same function starting from a point in the functions domain. From the Taylor's theorem \cite{apostol1964mathematical} the linear approximation of a function \(f\) which is differentiable at a point \(x_0\) is given by:
\begin{center}
$\Bar{f}(x)=f(x_0)+(x-x_0)f'(x_0)$    
\end{center}
The Newton Step \( \Delta x_n\) can be obtained by finding the root of this linear approximation of \(f\) as $f'(x_n) \Delta x_n=-f(x_n)$.In the matrix system the equivalent representation to obtain the newton step would be {\(J(x_n) \Delta x_n=-F(x_n)\)} where \(J(x_n)\) is the Jacobian of \(F(x)\) at \(x_n\). These newton steps are the search directions along which the the linear approximation of the function will be iteratively evaluated. The Newton Iterates are hoped to converge to the roots of the function \(F(x)\)\cite{apostol1964mathematical}. 

\section{Interior Point Methods and PDIPM's}\label{IPM}

\section{Path-Following Algorithms}\label{PFA}

\section{Mehrotra's Predictor Corrector Algorithm}
\section{Complexity Analysis of PDIPM's}

